\documentclass[12pt]{article}

\usepackage{fullpage}
\usepackage{amsmath}
\usepackage{amssymb} 
\usepackage{natbib}
\usepackage{url}

\title{EC 782 Assignment \#7 \\ The Effect of Increased Medicalization on Female Genital Mutilation as a Signal of ``Quality''}
\author{Karim Nagib}

\begin{document}

\maketitle

\section{Introduction}

Female genital mutilations (FGM) \footnote{This is sometimes referred to with the less judgemental terms like ``circumcision'' and ``cutting''. However, I opted to use the term used by the World Health Organization.} is the traditional practice in some cultures of removing the whole or part of the external female genitalia of girls, from as young as infancy to 15, for non-medical reasons. Some forms of the practice also seal the vaginal opening. It is mostly done as a rite of passage into female adulthood, an act of ``cleansing'' in preparation for marriage, and/or a method of curbing sexuality to ensure virginal purity before marriage and fidelity after\cite{whofs241}. It is prevalent to different degrees in western, eastern and northern Africa; its occurrence could be as high as 91\% of women between the ages of 15 and 49, in countries like Egypt\cite{whoprev, El-Zanat}. World-wide, between 100 to 140 million girls are estimated to have undergone this procedure \cite{whofs241}.

The FGM procedure has been found to cause a variety of health problems if carried out in an non-sanitary environment (by traditional circumcisers, for example), as well as long term problems and complications in childbirth.  No health benefits have been found. Additionally, it has been internationally recognized as a violation of the human rights of girls who are forced to undergo this procedure \cite{whofs241}. It has been compared to foot-binding in being a traditional practice harmful to society with no benefits, as well as being ethically indefensible due to its permanent physical and psychological damage \cite{Mackie1996}. The WHO has directed its advocacy and research efforts towards the elimination of this practice, in conjunction with local governments that have attempted a variety of policy interventions \cite{whofs241}. In addition to health-related or ethical objections to FGM, it has been found in a study conducted by the WHO on the financial cost of FGM in six African countries, that the obstetric complications caused by it to be \$ 3.7 million (PPP) \cite{Bishai2010}.

Typically, the procedure of FGM was done by a traditional circumciser, but that role is quickly being taken over by professional health providers, such as doctors or nurses.  This is referred to as the ``medicalization'' of FGM and it has become a major concern for the WHO and many anti-FGM activists. The WHO's declares that under no circumstances should health professionals preform FGM, regarding it as violation of the medical ethic of ``Do no harm''. There are also fears that medicalization might legitimize the practice, giving it the appearance of being beneficial, and hence rolling back the gains made in the elimination of FGM \cite{OHCHR}. A more amendable position views medicalization as a harm reducing temporary solution in societies where a sudden elimination of FGM is unlikely to take place.  Such a view, regards the resistance to medicalization as counterproductive and harmful to the young girls who would then have to suffer the painful procedure without anesthetics, and proper sanitation and care at the hands of professional health provider \cite{Shell-Duncan2001}. 

This paper investigates the possibility that medicalization might inadvertently have another side-effect on the practice of FGM.  It follows the argument that FGM is an investment for marriage; it is a form of ritualistic marking signaling the higher ``quality'' of potential wives in the marriage market. Thus, the question this paper is addressing is whether medicalization contributes to the breakdown of the traditional mechanism by which FGM is signaled in the marriage market.  In the context investigated, Egypt, it is argued that circumcision is effectively an unobservable and unverifiable characteristic that must rely on ceremony or a reliable ``certifier''.  As the procedure is increasingly done by health professionals in health clinics, and no longer by traditional circumcisers\footnote{In Egypt, mostly midwives (\emph{daya}) and barbers} in the child's or a relative's home, the process by which information about FGM is conveyed in society is weakened as health providers are less effective at play the role of FGM ``certifier'' or are simply unwilling to do so. In other words, medicalization, by breaking the traditional information network, may introduce friction to the marriage market that would reduce the incentive to invest in FGM---arguably a desirable ``underinvestment''.

This paper follows Becker's classical work on marriage markets \cite{Becker1981}.  This work is particularly important in studying low-income countries, where marriage is a critical aspect of a woman's life, in the face of low education and few independent employment opportunities outside the role of wife and mother. This paper was also inspired by Chesnokova and Vaithianathan's work on the persistence of FGM as an equilibrium in society. They find that as long there exist some circumcised women and circumcision is viewed as a desirable quality by men, there will always be an incentive to have FGM done to girls to ensure marriageability \cite{Chesnokova2007}. The paper also follows the literature on pre-marital investment \cite{Burdett2001, Peters2002}.  However, this work is more in terms of the undesirability of any investment in FGM.  

Finally, this work being focused on the signaling aspect of FGM, it is related to Rai's work on the signaling of parents to prospective suitors for their daughters by disciplining or confining them \cite{Raia}. 

\section{Female Genital Mutilation in Egypt}

\section{Model}

\bibliographystyle{named}
\bibliography{$HOME/Documents/library}

\end{document}
